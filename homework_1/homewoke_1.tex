\documentclass[10pt,a4paper]{article}
	% 使用12pt(对应于中文的小四号字)
	% \usepackage{xeCJK} 
	% \setmainfont{Times New Roman} 
	% \setCJKmainfont[BoldFont=Hei]{Hei} 
	\usepackage{ctex}
	\usepackage{geometry}   %设置页边距的宏包
    \usepackage{titlesec}   %设置页眉页脚的宏包
    \usepackage{amssymb}
    \usepackage{amsmath}
    \geometry{left=3.17cm,right=3.17cm,top=2.54cm,bottom=2.54cm}  %设置 上、左、下、右 页边距
	\begin{document}
	\newpagestyle{main}{            
		\sethead{哈尔滨工业大学}{数据库系统\_第一次作业}{1160300314 朱明彦}     %设置页眉
		% \setfoot{左页脚}{中页脚}{右页脚}      %设置页脚,可以在页脚添加 \thepage  显示页数
		\setfoot{}{}{\thepage}
		\headrule                                      % 添加页眉的下划线
		% \footrule                                       %添加页脚的下划线
	}
	\pagestyle{main}    %使用该style
	
	% \setlength{\baselineskip}{15.6pt}
	\setlength{\parskip}{0pt}
	\renewcommand{\baselinestretch}{1.5}
	
    \paragraph{一、}
    \subparagraph{1. A} 可以举出以下反例

    \begin{minipage}{\textwidth} 
        \begin{minipage}[t]{0.35\textwidth} 
        \centering 
        \makeatletter\def\@captype{table}\makeatother\caption{R} 
        \begin{tabular}{c|c}
            \hline
            A & F \\ \hline
            1 & 2 \\
            3 & 4 \\
            \hline
        \end{tabular} 
    \end{minipage} 
    \begin{minipage}[t]{0.35\textwidth} 
        \centering 
        \makeatletter\def\@captype{table}\makeatother\caption{S} 
        \begin{tabular}{c|c}
            \hline
            A & F \\ \hline
            2 & 2 \\
            3 & 2 \\
            \hline
        \end{tabular} 

    \end{minipage} 

    \end{minipage}
    \vspace{0.5cm}\\
    可以看到,对于左式$\Pi_F(R_1 \cap R_2) = \varnothing$;而对于右式$\Pi_F(R_1) \cap \Pi_F(R_2) = \{2\}$。因而A错误。

    \subparagraph{2. A} 对于R$\times$S的前r列为R的一个元组,后s列为S的一个元组,故原来比较中使用的为R的第i列和S的第j列,此处应该换为R$\times$S的第i列和第j+r列。
    \paragraph{二、}
    \subparagraph{1.} $\Pi_{SNO, PNO}(SPJ) \div \Pi_{SNO}(\sigma_{CITY = LONDON}(S))$
    \subparagraph{2.} $\Pi_{PNO}(SPJ \Join S \Join J)$
    \subparagraph{3.} $\Pi_{JNAME}(\sigma_{SNO = S1}(SPJ) \Join J)$
    \subparagraph{4.} $\Pi_{JNAME}((SPJ \Join J) \Join (\Pi_{PNO}(\sigma_{COLOR = red}(P))))$
    \subparagraph{5.} $\Pi_{JNAME, PNO}(SPJ \Join J) \div \Pi_{PNO}(P)$
    \subparagraph{6.} $\Pi_{SNAME, PNO}(SPJ \Join S) \div \sigma_{PNO = P1 \lor PNO = P2}(\Pi_{PNO}(P))$
    \subparagraph{7.} $\Pi_{PNAME}(P \Join \Pi_{COLOR}(\sigma_{PNAME = "TV"}(P)))$

    \paragraph{三、}
    \subparagraph{(1)} $\{t | \exists u \in R \land t[B] = u[B] \land t[C] = u[C]\}$
    \subparagraph{(2)} $\{t | t \in R \land t[B] > t[C]\}$
    \subparagraph{(3)} $\{t | t \in R \lor t \in S\}$
    \subparagraph{(4)} $\{t | t \in R \land t \in S\}$
    \subparagraph{(5)} $\{t | t \in R \land t \notin S\}$
    \subparagraph{(6)} $\{t | \exists r \in R \land \exists s \in S \land r[B] = s[B] \land t[A] = r[A] \land t[B] = r[B] \land t[C] = s[C]\}$
    \subparagraph{(7)}
    \begin{equation*}
        \begin{split}
            & \{t | t \in \{u | \exists r \in R \land u[A] = r[A] \land u[B] = r[B]\} \\
            &      \land (\forall s \in \{v | \exists p \in S \land v[C] = p[C]\})(\exists r \in R \land r[A] = t[A] \land r[B] = t[B] \land s[C] = r[C])\}
        \end{split}
    \end{equation*}

    \paragraph{四、}
    \begin{itemize}
    \item 联系:限制在安全表达式范围内的元组关系演算,限制在安全表达式范围内的域关系演算以及关系代数的表达能力等价。
    \item 区别:
     \begin{itemize}
        \item \textbf{关系代数}是一种抽象的查询语言,它用对关系的运算来表达查询;并且有限关系上的关系代数操作结果都不会导致无限关系和无穷验证,因此关系代数系统是安全的。
        \item \textbf{元组关系演算}是非过程化的查询语言,只描述所需信息,而不给出获得该信息的具体过程;并且元组关系演算系统不是安全的。
        \item \textbf{域关系演算}是另一种关系演算的另一种形式,使用从属性域中取值的域变量,而不是整个元组的值;并且关系演算系统也不是安全的。
    \end{itemize}
    \end{itemize}
    

\end{document}