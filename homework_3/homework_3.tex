\documentclass[10pt,a4paper]{article}
	% 使用12pt(对应于中文的小四号字)
	% \usepackage{xeCJK} 
	% \setmainfont{Times New Roman} 
	% \setCJKmainfont[BoldFont=Hei]{Hei} 
    \usepackage{array,ragged2e,pst-node,pst-dbicons}
	\usepackage{ctex}
	\usepackage{geometry}   %设置页边距的宏包
    \usepackage{titlesec}   %设置页眉页脚的宏包
    \usepackage{amssymb}
    \usepackage{amsmath}
    \usepackage{tikz-er2}
    \usepackage{graphviz}
    \geometry{left=3.17cm,right=3.17cm,top=2.54cm,bottom=2.54cm}  %设置 上、左、下、右 页边距
	\begin{document}
	\newpagestyle{main}{            
		\sethead{哈尔滨工业大学}{数据库系统\_第一次作业}{1160300314 朱明彦}     %设置页眉
		% \setfoot{左页脚}{中页脚}{右页脚}      %设置页脚,可以在页脚添加 \thepage  显示页数
		\setfoot{}{}{\thepage}
		\headrule                                      % 添加页眉的下划线
		% \footrule                                       %添加页脚的下划线
	}
	\pagestyle{main}    %使用该style
	\usetikzlibrary{positioning}
	\usetikzlibrary{shadows}
	\usetikzlibrary{arrows}
	\tikzstyle{every entity} = [top color=white, bottom color=blue!30, 
                            draw=blue!50!black!100, drop shadow]
	\tikzstyle{every weak entity} = [drop shadow={shadow xshift=.7ex, 
                                 shadow yshift=-.7ex}]
	\tikzstyle{every attribute} = [top color=white, bottom color=yellow!20, 
                               draw=yellow, node distance=1cm, drop shadow]
	\tikzstyle{every relationship} = [top color=white, bottom color=red!20, 
                                  draw=red!50!black!100, drop shadow]
	\tikzstyle{every isa} = [top color=white, bottom color=green!20, 
                         draw=green!50!black!100, drop shadow]

	% \setlength{\baselineskip}{15.6pt}
	\setlength{\parskip}{0pt}
	\renewcommand{\baselinestretch}{1.5}
	
	\paragraph{一、}
	\begin{enumerate}
		\item \textbf{实体}是ER模型中的基本对象,是现实世界中各种失误的抽象。
		\item \textbf{属性}是每个实体所拥有的一组特征或性质,实体的属性值是数据库中存储的主要数据。
		\item \textbf{联系}是不同实体集之间的某种关联,可以使用联系集定义也可以通过实体的属性定义。
	\end{enumerate}
	\paragraph{二、}
	\begin{center}
		\scalebox{0.75}{
			\begin{tikzpicture}[node distance=1.5cm, every edge/.style={link}]
		
				\node[entity] (customer) {顾客};
				\node[attribute] (cname) [above=of customer] {姓名} edge (customer);
				\node[attribute] (cid) [above left=of customer] {\key{社会安全号码}} edge (customer);

				\node[relationship] (have) [below=of customer] {拥有} edge node[auto] {1} (customer);
				\node[entity] (address) [below left=of have] {地址} edge node[auto] {m} (have);
				\node[attribute] (state) [left=of address] {\key{州名}} edge (address);
				\node[attribute] (city) [above left=of address] {\key{城市}} edge (address);
				\node[attribute] (street) [below left=of address] {\key{街道}} edge (address);

				\node[entity] (telephone) [below right=of have] {电话} edge node[auto] {n} (have);
				\node[attribute] (telephone_number) [right=of telephone] {\key{电话号码}} edge (telephone);
			  
				% \node[isa] (isa) [below=1cm of emp] {ISA} edge (emp);
			  
				% \node[entity] (mec) [below left=1cm of isa] {Mechanic} edge (isa);
				% \node[entity] (sal) [below right=1cm of isa] {Salesman} edge (isa);
			  
				% \node[relationship] (does) [left=of mec] {Does} edge (mec);
			  
				% \node[weak entity] (rep) [below=of does] {RepairJob} edge (does);
				% \node[attribute] (rnum) [left=of rep] {\discriminator{Number}} edge (rep);
				% \node[attribute] (desc) [above left=of rep] {Description} edge (rep);
				% \node[attribute] (cost) [below left=of rep] {Cost} edge (rep);
				% \node[attribute] (mat) [left=0.5cm of cost] {Parts} edge (cost);
				% \node[attribute] (work) [below left=0.5cm of cost] {Work} edge (cost);
			  
				% \node[ident relationship] (reps) [below=of rep] {Repairs} edge [total] (rep);
			  
				% \node[entity] (car) [right=of reps] {Car} edge [<-] (reps);
				% \node[attribute] (lic) [above=of car] {\key{License}} edge (car);
				% \node[attribute] (mod) [below=of car] {Model} edge (car);
				% \node[attribute] (year) [below right=of car] {Year} edge (car);
				% \node[attribute] (manu) [below left=1.5cm of car] {Manufacturer} edge (car);
				
				% \node[relationship] (buy) [below=of sal] {Buys};
				% \node[attribute] (pri) [above left=of buy] {Price} edge (buy);
				% \node[attribute] (sdate) [left=of buy] {Date} edge (buy);
				% \node[attribute] (bval) [below left=of buy] {Value} edge (buy);
			  
				% \node[relationship] (sel) [right=of buy] {Sells};
				% \node[attribute] (sdate) [above right=of sel] {Date} edge (sel);
				% \node[derived attribute] (sval) [right=of sel] {Value} edge (sel);
				% \node[attribute] (com) [below right=of sel] {Comission} edge (sel);
				
				% \draw[link] (car.10) -| (buy) (buy) edge (sal);
				% \draw[link] (car.-10) -| (sel) (sel) |- (sal);
			  
				% \node[entity] (cli) [below right=0.5cm and 3.7cm of car] {Client};
				% \node[attribute] (cid) [right=of cli] {\key{ID}} edge (cli);
				% \node[attribute] (cname) [below left=of cli] {Name} edge (cli);
				% \node[multi attribute] (cphone) [below right=of cli] {Phone} edge (cli);
				% \node[attribute] (cadd) [below=of cli] {Address} edge (cli);
			  
				% \draw[link] (cli.70) |- node [pos=0.05, auto, swap] {buyer} (sel);
				% \draw[link] (cli.110) |- node [pos=0.05, auto] {seller} (buy);
			  
			  \end{tikzpicture}	  
		}
	\end{center}
    % \paragraph{四、}
    % \paragraph{三、}
    

\end{document}