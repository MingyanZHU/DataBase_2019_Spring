\documentclass{ML}

% 姓名,学号
\infoauthor{朱明彦}{1160300314}

% 课程类型,实验名称
\infoexp{课程类型}{概述}

\infoschool{计算机学院}{高宏}

\begin{document}
\maketitle

\tableofcontents
\newpage

\begin{center}
    \textbf{\zihao{3} 论文概述}
\end{center}

\section{解决的问题}
选择的论文\cite{toain}为2018年PVLDB上面的一篇文章,\textbf{主要解决的问题是Road Networks上面动态的kNN查询}。

针对这个问题,作者主要有如下\textbf{三个方面}的工作:
\begin{enumerate}
    \item 对于Road Networks上需要进行kNN查询的系统建立新的数学模型,并找到其中影响系统整体吞吐量的关键因素。
    \item 建立了一个以Shortcut Graph为基础的索引,SCOB。
    \item 设计了可以调整SCOB以最大化系统吞吐量的算法,TOAIN。
\end{enumerate}
\section{采用的思想}
\subsection{数学模型方面}
\subsection{索引SCOB方面}
\subsection{算法TOAIN方面}
\section{基本算法描述}
文中以伪代码形式讲述的算法共有4个,下面将分别进行描述。
\subsubsection{SCOB Index}
\paragraph{Query}
\paragraph{Insert}
\paragraph{Delete}
\subsubsection{TOAIN}
\paragraph{Compute Rank}
\section{算法分析}

\section{举例说明}

\appendix

% \section{源代码}
% \section{参考文献}
\begin{thebibliography}{20}
    \bibitem{toain} Luo, S., Kao, B., Li, G., Hu, J., Cheng, R., \& Zheng, Y. (2018). TOAIN: a throughput optimizing adaptive index for answering dynamic k NN queries on road networks. Proceedings of the VLDB Endowment, 11(5), 594-606.
\end{thebibliography}

\end{document}
