\documentclass{ML}

% 姓名,学号
\infoauthor{朱明彦}{1160300314}

% 课程类型,实验名称
\infoexp{课程类型}{实验一}

\infoschool{计算机学院}{高宏}

\begin{document}
\maketitle

% \tableofcontents
\newpage

\begin{center}
    \textbf{\zihao{3} 实验一 MySQL关系数据库管理系统及SQL语言的使用}
\end{center}

\section{实验目的}
    掌握 MySQL 关系数据库管理系统的基本命令,并熟练使用 SQL 语言管理
MySQL 数据库。掌握 SQL 语言的使用方法,学会使用 SQL 语言进行关系数据
库查询,特别是聚集查询、连接查询和嵌套查询。
\section{实验环境}
\begin{itemize}
    \item Ubuntu 16.04.5
    \item MySQL Ver 14.14 Distrib 5.7.25
\end{itemize}
\section{实验过程及结果}
对报告中第 3 部分的实验结果进行截图,作为评分标准。对于 3.2,使用
MYSQL 基本命令可以查看创建的关系数据库的每个模式(参看第 4 部分 MYSQL
手册,做作业之前先练习第 4 部分有助于熟悉 MYSQL 命令),对结果进行截图。
对于 3.3,使用 select *语句分别显示每张表中被添加的数据,对结果进行截图。
对于 3.1,在你完成 3.2 和 3.3 之后,把输入的 9 条 sql 语句和相应的查询结果截
图显示,所以你设计的数据库必须包括 3.1 需要的各种数据。
\section{实验心得}

% \appendix

% \section{源代码}


\end{document}
